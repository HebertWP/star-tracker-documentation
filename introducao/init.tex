\chapter{Introdução}
\label{cap:Introducao_init}


Um dos principais sistemas de um satélite é o sistema de determinação de atitude, o qual é parte do sistema de controle de atitude, que é responsável pela orientação espacial do satélite. 
Este sistema é de extrema importância para os satélites, pois, na maioria das suas aplicações, é requerido que as antenas ou câmeras do dispositivos apontem para regiões específicas da Terra. 
Como para tirar fotos da superfície terrestre e fornecer acesso de rede a estação fixa em solo, como por exemplo, é o caso do Starlink.

Durante este projeto, será desenvolvido sistema de rastreamento estrelar para cubesat, com aplicabilidade  na indústria aeroespacial. O sistema será baseado em visão computacional. Pretende-se utilizar webcams convencionais e utilizar como unidade de processamento embarcados de baixo custo que executam Linux, como por exemplo a Raspberry Pi 4.

O foco desse projeto será no desenvolvimento dos algoritmos responsáveis por fazer a captura e análise das imagens, e determinar a posição  angular do cubesat. Para testar o sistema, será desenvolvido uma simulação. Ela consistirá de um monitor juntamente de um software  de simulação do céu estrelado a ser visualizado pelo dispositivo. 

Essa simulação, será desenvolvida em linguagem Python e permitirá a rotação em todo o espaço com 360 graus de liberdade em todos os eixos, por fim será realizada uma validação final com uma webcam.
