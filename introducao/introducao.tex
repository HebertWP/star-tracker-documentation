
\chapter{Introdução}
\label{cap:Introducao_init}


Um dos principais sistemas de um satélite é o sistema de determinação de atitude, o qual é parte do sistema de controle de atitude, que é responsável pela orientação espacial do satélite. 
Este sistema é de extrema importância para os satélites, pois, na maioria das suas aplicações, é requerido que as antenas ou câmeras do dispositivos apontem para regiões específicas da Terra. 
Como para tirar fotos da superfície terrestre e fornecer acesso de rede a estação fixa em solo, como por exemplo, é o caso do Starlink.

Durante este projeto, será desenvolvido sistema de rastreamento estrelar para cubesat, com aplicabilidade  na indústria aeroespacial. O sistema será baseado em visão computacional. Pretende-se utilizar webcams convencionais e utilizar como unidade de processamento embarcados de baixo custo que executam Linux, como por exemplo a Raspberry Pi 4.

O foco desse projeto será no desenvolvimento dos algoritmos responsáveis por fazer a captura e análise das imagens, e determinar a posição  angular do cubesat. Para testar o sistema, será desenvolvido uma simulação. Ela consistirá de um monitor juntamente de um software  de simulação do céu estrelado a ser visualizado pelo dispositivo. 

Essa simulação, será desenvolvida em linguagem Python e permitirá a rotação em todo o espaço com 360 graus de liberdade em todos os eixos, por fim será realizada uma validação final com uma webcam.


\section{Motivação}
\label{sec:Introducao_motivacao}
Com o aprimoramento da eletrônica, os circuitos e sistemas presentes em satélites se tornaram menores, mais leves, mais baratos, rápidos, e com maior eficiência  energética. Além disso, o desenvolvimento de uma padronização nas dimensões destes pequenos satélites possibilitou um decréscimo ainda maior de custo.

Em universidades e  \textit{StartUps} o estudo e desenvolvimento de cubesats vêm crescendo rapidamente, mesmo que os pequenos satélites apresentem limitações físicas e energéticas, o custo benefício em sua aplicabilidade é grande.

Outra limitação está na capacidade do satélite de se localizar e orientar no espaço, ou seja, controlar a sua atitude, que devido às restrições já mencionadas, costumam ser extremamente limitados ou mesmo inexistentes ~\cite[]{Diaz}. Com isto, as possibilidades de aplicações destes cubesats tornam-se consideravelmente limitadas.

A primeira etapa para realização do controle de atitude, é identificar de forma confiável, precisa e contínua, 
a orientação do satélite ~\cite[]{Diaz}. No espaço existem pontos de referência que podem ser utilizados para a determinação da atitude, 
como o Sol, Lua e a Terra, porém estas referências não são consistentes, já que o Sol pode estar encoberto, 
e a análise da superfície terrestre vista do espaço varia muito, devido a nuvens e outros fenômenos meteorológicos. 
Além disso, a análise de imagens complexas é custosa computacionalmente, 
devido às limitações de volume e energia, tornam a aplicação extremamente complicada.

Outra opção é utilizar o campo magnético da Terra, porém, a interferência eletromagnética é algo relativamente comum, 
uma vez que os próprios circuitos elétricos do satélite podem gerar interferências, 
o que inviabiliza a utilização dessa grandeza física para monitoramento de atitude de satélites.

Uma terceira opção é a utilização de uma câmera realizando a análise das estrelas, o fato do satélite estar no espaço faz com que as estrelas estejam na maioria do tempo no campo de visão do satélite ~\cite[]{Tappe}.

Realizar o controle de atitude apenas utilizando IMU (\textit{Inertial Measurement Unit})  é difícil, pois IMU são suscetíveis a erros de desvio de Offset, erros de Instabilidade, temperatura, são sensíveis a pancadas e vibrações ~\cite[]{Young}. 
Desta forma, utiliza-se IMUs de alto custo, m sua maioria baseada em transdutores ópticos, como os IFOGs (\textit{interferometric fiber optic gyroscopes}) e os RLGs (\textit{ring laser gyros}), 
que são caros, e em sua maioria, grandes. 

Apesar de IMUs de baixo custo não serem adequadas para identificação de atitute através da integração das velocidades angulares dos girsocópios, 
que devido a deriva relativamente elevada, resulta em erros de orientação consideráveis em poucos segundo, 
podem ser utilizados em conjunto com outras abordagens, como sensores de estrelas, para melhorar a acurácia e/ou aumentar a taxa de atualização através de fusão sensorial.

Devido a questões de custo e disponibilidade, para aplicação em cubsat dar-se preferência para dispositivos de prateleira (sensores e chips já prontos e vendidos em massa). 
Geralmente fazendo uso da tecnologia MEMS (\textit{Micro Electro Mechanical Systems}), os quais são relativamente baratos, pequenos e possuem massa reduzida. 


\section{Objetivos}
\label{sec:Introducao_objectives}

O objetivo deste trabalho é realizar o desenvolvimento de um sistema de rastreamento estrelar, que utiliza uma camera comercial para detectar estrelas, 
aliado a um algoritmo de rastreamento estrelar, que seja capaz de rodar em um computador de bordo, com baixo custo, baixo consumo de energia e com volume mássico e peso reduzido.

O desenvolvimento de tal sistema complexo exige a elaboração de multiplos sistemas auxiliares, como o desenvolvimento da simulação interamente virtual do sistema, 
que é necessario para se realizar o testes dos algoritmos de rastreamento estrelar desenvolvidos. 
Após esta etapa o sistema é testado em simuladores com hardware real, para se verificar a viabilidade do sistema, e então o sistema é testado em um protótipo real.
Devido a restrições de tempo neste trabalho é desenvolvido e testado apenas para o primeiro tipo de simulação, a simulação virtual.


\section{Estrutura}
\label{sec:Introducao_structure}

Este trabalho esta organizado da seguinte forma


%\section{Determinação da Atitude}
%\label{sec:sec01}
%Atitude de um satélite consiste em sua orientação no espaço, um satélite possui 6 graus de liberdade, (três de rotação e três de translação)  normalmente  modelados em 12 variáveis de estado: 6 variáveis de posição e 6 de suas respectivas derivadas, já que a estabilização e o controle de orientação são normalmente desejados no controle ~\cite[]{Souza}.

%Do Cálculo sabemos, que a derivada da posição a sua velocidade, e sabendo-se o valor da posição a todo instante pode-se calcular a derivada, portanto é possível facilmente obter a informação da velocidade. Desta forma, é necessário apenas a utilização de sensores de posição, o que é verdade, porém a precisão se eleva muito ao utilizarmos múltiplos sensores, e realizarmos uma fusão sensorial completa. 

%Este trabalho foca no desenvolvimento do rastreador estelar que é responsável pela determinação da rotação nos 3 eixos, já pensado para a incorporação em tais conjuntos de dados futuramente.

