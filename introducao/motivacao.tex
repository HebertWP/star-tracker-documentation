\section{Motivação}
\label{sec:Introducao_motivacao}
Com o aprimoramento da eletrônica, os circuitos e sistemas presentes em satélites se tornaram menores, mais leves, mais baratos, rápidos, e com maior eficiência  energética. Além disso, o desenvolvimento de uma padronização nas dimensões destes pequenos satélites possibilitou um decréscimo ainda maior de custo.

Em universidades e  \textit{StartUps} o estudo e desenvolvimento de cubesats vêm crescendo rapidamente, mesmo que os pequenos satélites apresentem limitações físicas e energéticas, o custo benefício em sua aplicabilidade é grande.

Outra limitação está na capacidade do satélite de se localizar e orientar no espaço, ou seja, controlar a sua atitude, que devido às restrições já mencionadas, costumam ser extremamente limitados ou mesmo inexistentes ~\cite[]{Diaz}. Com isto, as possibilidades de aplicações destes cubesats tornam-se consideravelmente limitadas.

A primeira etapa para realização do controle de atitude, é identificar de forma confiável, precisa e contínua, 
a orientação do satélite ~\cite[]{Diaz}. No espaço existem pontos de referência que podem ser utilizados para a determinação da atitude, 
como o Sol, Lua e a Terra, porém estas referências não são consistentes, já que o Sol pode estar encoberto, 
e a análise da superfície terrestre vista do espaço varia muito, devido a nuvens e outros fenômenos meteorológicos. 
Além disso, a análise de imagens complexas é custosa computacionalmente, 
devido às limitações de volume e energia, tornam a aplicação extremamente complicada.

Outra opção é utilizar o campo magnético da Terra, porém, a interferência eletromagnética é algo relativamente comum, 
uma vez que os próprios circuitos elétricos do satélite podem gerar interferências, 
o que inviabiliza a utilização dessa grandeza física para monitoramento de atitude de satélites.

Uma terceira opção é a utilização de uma câmera realizando a análise das estrelas, o fato do satélite estar no espaço faz com que as estrelas estejam na maioria do tempo no campo de visão do satélite ~\cite[]{Tappe}.

Realizar o controle de atitude apenas utilizando IMU (\textit{Inertial Measurement Unit})  é difícil, pois IMU são suscetíveis a erros de desvio de Offset, erros de Instabilidade, temperatura, são sensíveis a pancadas e vibrações ~\cite[]{Young}. 
Desta forma, utiliza-se IMUs de alto custo, m sua maioria baseada em transdutores ópticos, como os IFOGs (\textit{interferometric fiber optic gyroscopes}) e os RLGs (\textit{ring laser gyros}), 
que são caros, e em sua maioria, grandes. 

Apesar de IMUs de baixo custo não serem adequadas para identificação de atitute através da integração das velocidades angulares dos girsocópios, 
que devido a deriva relativamente elevada, resulta em erros de orientação consideráveis em poucos segundo, 
podem ser utilizados em conjunto com outras abordagens, como sensores de estrelas, para melhorar a acurácia e/ou aumentar a taxa de atualização através de fusão sensorial.

Devido a questões de custo e disponibilidade, para aplicação em cubsat dar-se preferência para dispositivos de prateleira (sensores e chips já prontos e vendidos em massa). 
Geralmente fazendo uso da tecnologia MEMS (\textit{Micro Electro Mechanical Systems}), os quais são relativamente baratos, pequenos e possuem massa reduzida. 
