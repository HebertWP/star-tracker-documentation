\section{Desenvolvimento de software}

\subsection{\textit{Test Driven Development} (TDD)}

O  Desenvolvimento Orientado a Testes, consiste em desenvolver testes separados para cada parte de um software como é o caso das funções, os testes devem garantir que o código funcione e cumpra a tarefa desejada.

Para aplicá-lo, primeiro cria-se o teste, dessa forma garante-se que o que durante o desenvolvimento do código o resultado inicialmente esperado seja obtido, dessa forma o teste falha na primeira vez que é executado, já que este está testando o recurso que ainda não existe, em seguida é desenvolvido o recurso de forma a fazer o teste passar, então repete-se este ciclo função por função. 

Caso seja localizado um \textit{bug} no software, 
primeiramente o desenvolvedor deve implementar um teste que consiga replicar o erro em específico, só então a realizar a correção, dessa forma se garante a correta implementação da solução, assim como o novo software deve manter-se passando nos testes anteriores. Algumas das vantagem de tal método são:
\begin{itemize}
    \item Facilidade em focar em problemas específicos de desenvolvimento;
    \item Códigos são fáceis de refatorar;
    \item Facilidade em corrigir bugs, mesmo durante o desenvolvimento;
    \item Capacidade em garantir do que está funcionando e como está funcionando;
    \item Dividir de forma mais clara cada parte do código; 
    \item Descobrir falhas de forma prematura durante o processo de desenvolvimento;
    \item Maior organização de código;
\end{itemize}

Como este projeto é desenvolvido em python, 
é utilizado o framework \textit{pytest} que realiza e execução de testes em cada parte do código separadamente, 
neste trabalho, apenas os módulos do software com as implementações matemáticas, 
foram testados por tal metodologia.
Os códigos relacionados a interfaces gráficas não foram desenvolvidos dessa forma, 
isto ocorreu pois, apesar de ser possível a utilização TDD, 
os testes para tails componentes de software são de criação lenta e difícil, 
optando-se então por apenas realizar \textit{checks} visuais.

\subsection{\textit{Model View Controller} (MVC)}

MVC é um arquitetura de software desenvolvida para otimizar o processo de desenvolvimento de interfaces gráficas, 
com foco em diminuir o tempo de manutenção do código, 
facilitando a depuração de elementos visuais do sistema, 
também facilitando a interação de múltiplos desenvolvedores e proporciona uma maior clareza de código exigindo um baixo nível de acoplamento entre as partes. 

Para isso, o MVC se divide em 3 componentes principais, o \textit{model}, o \textit{controller} e o \textit{view}. 
As regras de negócio estão contidas no model assim como as informações necessárias para execução do programa o \textit{controller} realizar a interface entre o \textit{model} e \textit{view}, 
que contém as especificações de como os usuários interagem com o sistema.

O \textit{view} é responsável por decide como apresentar as informações, 
por exemplo se uma lista de dados sera apresentada em forma de tabela ou gráfico,
ou inputs de informações, que pode ser linhas de textos, botoes, \textit{check box} etc.

Quando o usuário interage com este \textit{viwer}, 
é gerado um evento que é tratado pelo controlador,
que se preciso for irá formatar os dados gerados pela interface, 
então o controlador transfere o dado para o model, 
que executa a aplicação em si, 
como por exemplo realizando equações matemáticas, 
carregando arquivos e demais lógicas de negócio.

O \textit{model} também é responsável por armazenar as informações sobre o estado de execução do sistema.

No simulador estrelar aqui desenvolvido utiliza-se 3 \textit{views} separados, para gerenciar diferentes janelas de usuário, com um único modelo que é responsável por armazenar a lógica de negócios, que consiste na execução dos códigos de simulação.
