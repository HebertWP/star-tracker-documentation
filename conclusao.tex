\chapter*[Conclusão]{Conclusão}
\label{cap:Conclusao_init}
\addcontentsline{toc}{chapter}{Conclusão}

Nesta seção são sumarizados os principais resultados obtidos com esse projeto, 
juntamente com as principais conclusões e perspectivas futuras, e pontos a serem melhorados, 
o que envolve melhorias e adições ao Star Tracker em si, 
mas também envolvem a integração do sistema aqui desenvolvido com outros outros componentes do sistema de orientação espacial do cubesat.

Além disso, alguns erros de arquitetura do sistema cometidos pelo autor, não permitiram a analise com grandes conjuntos de dados, 
com diferentes frames, porém o sistema se mostra promissor.

\section*{Trabalhos futuros}
\begin{itemize}
	\item \textbf{Melhorias na implementação do algoritmo de correção de distorções}: Apesar da correção da distorção de pixel parecer aceitável, 
	pelos resultados apresentados na seção anterior, a incidência de erros na resposta final de detecção de estrelas, indica que esta etapas pode conter erros. 
	Uma das possíveis soluções, é a utilização da biblioteca open CV para realizar esta etapa, pois ela já possui algorítimos para resolução de distorções. 
	Optou-se inicialmente não utilizada por conter uma serie de recursos matemáticos ainda não completamente entendidos pelo autor. 
	\item \textbf{Utilização do triângulos esféricos}: Neste trabalho utiliza-se triângulos planares, porém a utilização de triângulos esféricos aumenta a precisão do sistema, 
	como é explicado por Cole ~\cite{Cole_2}.
	\item \textbf{Refatoração da estrutura de testes}: A atual estrutura de testes é pensada para ser de fácil compressão, 
	e permitir o teste de diferentes senários pelo usuário. 
	Porém o ideal séria uma estrutura de testes que permitisse a realização de testes em massa, 
	para que seja feita uma analise estatista como é realizada por ~\cite{Cole_2}, 
	o que não é possível com a estrutura atual.  
	Isto não se deve a limitações de algorítimo, 
	pois o algoritmo é capaz de realizar a detecção em tempo pequeno, 
	assim como pode ser executado em múltiplos processos em paralelo, 
	com a limitação sendo um erro estrutural de organização de testes. 
	Portanto, a refatoração destes códigos se faz necessária para um melhor desenvolvimento de etapas futuras. 
\end{itemize}

\section*{Perspectivas Futuras}

\begin{itemize}
	\item \textbf{Desenvolvimento de um simulador para teste de hardware}: Esta etapa já foi descrita na Seção ~\ref{cap:simulador}.
	\item \textbf{Construção física de um sistema de câmera}: A construção de um sistema físico, para deste mais fieis a realidade também se faz necessária, 
	nota-se que ao longo deste trabalho todos os códigos foram testado em Raspberry pi 4, o que facilita muito no desenvolvimento de etapas futuras. 
	\item \textbf{Integração com filtro de Kalman}: Com o protótipo físico em mãos pode-se iniciar a integração com outros sistema de medição para se refinar a precisão.
\end{itemize}