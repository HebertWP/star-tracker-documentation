\chapter*[Conclusão]{Conclusão}
\label{cap:Conclusao_init}
\addcontentsline{toc}{chapter}{Conclusão}


Ao longo do desenvolvimento deste trabalho, vários pontos de melhorias fora notados, 
assim como pontos de desenvolvimento futuros, o que envolve melhorias e adições ao Star Tracker em si, 
mas também envolvem a integração do sistema aqui desenvolvido com outros outros componentes do sistema de orientação espacial do cubesat,
é listado uma series de pontos a serem trabalhados.

Além disto alguns erros de arquitetura do sistema cometidos pelo autor, não permitiram a analise com grandes conjuntos de dados, 
com diferentes frames, porém o sistema se mostra promissor.

\section*{Correções e testes adicionais}
\begin{itemize}
	\item \textbf{Melhorias na implementação do algoritmo de correção de distorções}: Apesar da correção da distorção de pixel parecer aceitável, 
	pelos resultados apresentados na seção anterior, a incidência de erros na resposta final de detecção de estrelas, indica que esta etapas pode conter erros, 
	uma das possíveis soluções, é a utilização da biblioteca open CV para realizar esta etapa, pois esta já possuir algorítimos para resolução de distorções, 
	porém houve uma escolha de não ha utilizar pois está implementa uma serie de recursos matemáticos ainda não completamente entendidos pelo autor. 
	\item \textbf{Utilização do triângulos esféricos}: Neste trabalho utiliza-se triângulos planares, porém a utilização de triângulos esféricos aumenta a precisão do sistema, 
	como é explicado por Cole ~\cite{Cole_2}
	\item \textbf{Refatoração da estrutura de testes}: A atual estrutura de testes é pensada para ser de fácil compressão, 
	e permitir o teste de diferentes senários pelo usuário, porém o deste em milhares de entradas para se fazer analises estatísticas melhores não é possível de forma fácil, 
	isto não se deve a limitações de algorítimo, pois o algoritmo ao al do tipo, mas sim um erro estrutural de organização de testes.
	Portanto a refatoração destes códigos se faz necessária para um melhor desenvolvimento de etapas futuras. 
\end{itemize}

\section*{Perspectivas Futuras}

\begin{itemize}
	\item \textbf{Desenvolvimento de um simulador para teste de hardware}: Esta etapa já foi descrita na seção ~\ref{cap:simulador}
	\item \textbf{Construção física de um sistema de câmera}: A construção de um sistema físico, para deste mais fieis a realidade também se faz necessária, 
	nota-se que ao longo deste trabalho todos os códigos foram testado em raspberry pi 4, o que facilita muito no desenvolvimento ode futuras etapas. 
	\item \textbf{Integração com filtro dde Kalman}: Com o protótipo físico em mão pode se iniciar a integração com outros sistema de metição para se refinar a precisão.
\end{itemize}